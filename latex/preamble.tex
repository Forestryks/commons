\usepackage{cmap}
\usepackage{mathtext}
\usepackage[T2A]{fontenc}
\usepackage[utf8]{inputenc}
\usepackage[english,russian]{babel}
\usepackage{indentfirst}
\usepackage[unicode=true]{hyperref}
\usepackage{yfonts}
\usepackage{bbm}
\usepackage{blkarray}
\frenchspacing

\renewcommand{\epsilon}{\ensuremath{\varepsilon}}
\renewcommand{\succeq}{\ensuremath{\succcurlyeq}}
\renewcommand{\preceq}{\ensuremath{\preccurlyeq}}
\renewcommand{\phi}{\ensuremath{\varphi}}
\renewcommand{\kappa}{\ensuremath{\varkappa}}
\renewcommand{\le}{\ensuremath{\leqslant}}
\renewcommand{\leq}{\ensuremath{\leqslant}}
\renewcommand{\ge}{\ensuremath{\geqslant}}
\renewcommand{\geq}{\ensuremath{\geqslant}}
\renewcommand{\emptyset}{\varnothing}
\renewcommand{\a}{\ensuremath{\alpha}}
\renewcommand{\b}{\ensuremath{\beta}}
\renewcommand{\d}{\ensuremath{\delta}}
\newcommand{\D}{\ensuremath{\Delta}}
\newcommand{\g}{\ensuremath{\gamma}}
\renewcommand{\l}{\ensuremath{\lambda}}
\newcommand{\p}{\ensuremath{\phi}}
\newcommand{\s}{\ensuremath{\sigma}}
\newcommand{\eps}{\ensuremath{\epsilon}}
\newcommand{\ceil}[1]{\left\lceil #1 \right\rceil}
\newcommand{\floor}[1]{\left\lfloor #1 \right\rfloor}
\newcommand{\R}{\ensuremath{\mathbb{R}}}
\newcommand{\N}{\ensuremath{\mathbb{N}}}
\newcommand{\Q}{\ensuremath{\mathbb{Q}}}
\newcommand{\Z}{\ensuremath{\mathbb{Z}}}
\renewcommand{\C}{\ensuremath{\mathds{C}}}
\renewcommand{\O}{\ensuremath{\mathds{O}}}
\newcommand{\con}{\wedge}
\newcommand{\dis}{\vee}
\newcommand{\impl}{\to}
\newcommand{\mb}[1]{\mathbbm{#1}}
\newcommand{\false}{\text{Л}}
\newcommand{\true}{\text{И}}
\newcommand{\id}{\text{id}}
\newcommand{\ma}[1]{\begin{pmatrix} #1 \end{pmatrix}}
\newcommand{\mne}{M^0}

\DeclareRobustCommand{\divby}{%
  \mathrel{\vbox{\baselineskip.65ex\lineskiplimit0pt\hbox{.}\hbox{.}\hbox{.}}}%
}

\usepackage{amsmath,amsfonts,amssymb,amsthm,mathtools}
\usepackage{icomma}

\mathtoolsset{showonlyrefs=true}

\usepackage{euscript}
\usepackage{mathrsfs}

\let\ker\relax
\DeclareMathOperator{\sgn}{\mathop{sgn}}
\DeclareMathOperator{\re}{\mathop{Re}}
\DeclareMathOperator{\im}{\mathop{Im}}
\DeclareMathOperator{\rk}{\mathop{rk}}
\DeclareMathOperator{\ker}{\mathop{Ker}}
\DeclareMathOperator{\ord}{\mathop{ord}}
\DeclareMathOperator{\lcm}{\mathop{lcm}}
\DeclareMathOperator{\spec}{\mathop{Spec}}

\newcommand*{\hm}[1]{#1\nobreak\discretionary{}
{\hbox{$\mathsurround=0pt #1$}}{}}

\usepackage{graphicx}
\graphicspath{{images/}}
\setlength\fboxsep{3pt}
\setlength\fboxrule{1pt}
\usepackage{wrapfig}

\usepackage{array,tabularx,tabulary,booktabs}
\usepackage{longtable}
\usepackage{multirow}

\usepackage{minted}
\usepackage{xcolor}
\usepackage{dsfont}
\usepackage{enumitem}
\usepackage{eso-pic}

\renewcommand\qedsymbol{$\blacksquare$}

\makeatletter
\newcommand{\declprob}[2]{%
    \hyperlink{#1}{#2}%
    \protected@write\@mainaux{}{%
        \string\expandafter\string\gdef
            \string\csname\string\detokenize{#1}\string\endcsname{#2}%
    }
}
\newcommand{\useprob}[1]{%
    \hypertarget{#1}{\csname #1\endcsname}%
}
\makeatother

\theoremstyle{plain}
\newtheorem{theorem}{Теорема}
\newtheorem{proposition}[theorem]{Утверждение}
\newtheorem{suggestion}{Предложение}

\theoremstyle{definition}
\newtheorem{lemma}[theorem]{Лемма}
\newtheorem{corollary}{Следствие}[theorem]
\newtheorem{definition}{Определение}[section]

\newtheoremstyle{solution}
    {\dimexpr\topsep/2\relax} % space above
    {\dimexpr\topsep/2\relax} % space below
    {}          % body font
    {}          % indent amount
    {\itshape} % theorem head font
    {.}         % punctuation after theorem head
    {.5em}      % space after theorem head
    {#1}          % theorem hed spec. (empty = "normal")

\theoremstyle{solution}
\newtheorem{solution}{Решение}

\theoremstyle{remark}
\newtheorem*{note}{Замечание}

%\newenvironment{hproblem}[1]{
%\noindent\textbf{Задача #1.}%
%}{\vspace{1cm}}

\newtheoremstyle{problemstyle}
        {3pt} % <space above>
        {20pt} % <space below>
        {\normalfont} % <body font>
        {} % <indent amount>
        {\bfseries} % <theorem head font>
        {\normalfont\bfseries.} % <punctuation after theorem head>
        {.5em} % <space after theorem head>
        {#1 \useprob{#3}} % <theorem head spec (can be left empty, meaning `normal')>
\theoremstyle{problemstyle}
\newtheorem{problem}{Задача}
\theoremstyle{plain}

\newtheoremstyle{sproblemstyle}
        {3pt} % <space above>
        {20pt} % <space below>
        {\normalfont} % <body font>
        {} % <indent amount>
        {\bfseries} % <theorem head font>
        {\normalfont\bfseries.} % <punctuation after theorem head>
        {.5em} % <space after theorem head>
        {#1 #2} % <theorem head spec (can be left empty, meaning `normal')>
\theoremstyle{sproblemstyle}
\newtheorem{sproblem}{Задача}
\theoremstyle{plain}

\usepackage{extsizes} % Возможность сделать 14-й шрифт
\usepackage{geometry} % Простой способ задавать поля
\geometry{top=25mm}
\geometry{bottom=35mm}
\geometry{left=35mm}
\geometry{right=20mm}

\usepackage{lastpage}
\usepackage{soul}

\usepackage{csquotes}
\usepackage{multicol}


% \usepackage{tikz}
% \usepackage{pgfplots}
% \usepackage{pgfplotstable}
% \usepackage{color}
% \usepackage{pgf,tikz,pgfplots}
% \pgfplotsset{compat=1.15}
% \usepackage{mathrsfs}
% \usetikzlibrary{arrows}

\makeatletter
\newcommand{\DeclareMathActive}[2]{%
  % #1 is the character, #2 is the definition
  \expandafter\edef\csname keep@#1@code\endcsname{\mathchar\the\mathcode`#1 }
  \begingroup\lccode`~=`#1\relax
  \lowercase{\endgroup\def~}{#2}%
  \AtBeginDocument{\mathcode`#1="8000 }%
}

\newcommand{\std}[1]{\csname keep@#1@code\endcsname}
\patchcmd{\newmcodes@}{\mathcode`\-\relax}{\std@minuscode\relax}{}{\ddt}
\AtBeginDocument{\edef\std@minuscode{\the\mathcode`-}}
\makeatother

\DeclareMathActive{*}{\cdot}

\makeatletter
\newenvironment{sqcases}{%
  \matrix@check\sqcases\env@sqcases
}{%
  \endarray\right.%
}
\def\env@sqcases{%
  \let\@ifnextchar\new@ifnextchar
  \left\lbrack
  \def\arraystretch{1.2}%
  \array{@{}l@{\quad}l@{}}%
}
\makeatother

\DeclareMathOperator{\tr}{tr}
\DeclareMathOperator{\diag}{diag}
\DeclareMathOperator{\mat}{Mat}
\DeclareMathOperator{\elem}{\text{Э}}
\DeclareMathOperator{\Carg}{Arg}
\DeclareMathOperator{\carg}{arg}
\DeclareMathOperator{\range}{Range}
\DeclareMathOperator{\dom}{Dom}
\DeclareMathOperator{\Hom}{Hom}

\makeatletter
\renewcommand*\env@matrix[1][*\c@MaxMatrixCols c]{%
  \hskip -\arraycolsep
  \let\@ifnextchar\new@ifnextchar
  \array{#1}}
\makeatother

\makeatletter
\newcommand{\WronglyDeclarePairedDelimiter}[3]{%
  \expandafter\DeclarePairedDelimiter\csname RIGHT\string#1\endcsname{#2}{#3}%
  \newcommand#1{%
    \@ifstar{\csname RIGHT\string#1\endcsname}
            {\@ifnextchar[{\csname RIGHT\string#1\endcsname}
                          {\csname RIGHT\string#1\endcsname*}%
            }%
  }%
}
\makeatother

\WronglyDeclarePairedDelimiter\br{(}{)}
\WronglyDeclarePairedDelimiter\abs{\lvert}{\rvert}
\WronglyDeclarePairedDelimiter\set{\{}{\}}
\WronglyDeclarePairedDelimiter\ang{\langle}{\rangle}

\DeclareMathOperator*\lowlim{\underline{lim}}
\DeclareMathOperator*\uplim{\overline{lim}}

\DeclareRobustCommand{\divby}{%
  \mathrel{\text{\vbox{\baselineskip.65ex\lineskiplimit0pt\hbox{.}\hbox{.}\hbox{.}}}}%
}

\usepackage{listings}% http://ctan.org/pkg/listings
\lstset{
  basicstyle=\ttfamily,
  mathescape
}

\newcommand*\xor{\oplus}
\newcommand*\iso{\xrightarrow{\sim}}
\newcommand{\enum}[1]{\begin{enumerate}[noitemsep,topsep=0pt] #1 \end{enumerate}}
